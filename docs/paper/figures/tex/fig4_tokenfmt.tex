\begin{figure} 
	\centering
	\hspace{-1em}
	\begin{minipage}{5in}
		\footnotesize
		% could not figure out how to make lines break a colorbox, so that's why these line breaks are manual
\colorbox[RGB]{ 217,210,233 }{ \texttt{ <ADJLIST\_START> (0,0) <--> (1,0) ; (2,0) <--> (3,0) ; (4,1) <--> (4,0) ; (2,0) <--> (2,1) ; }} 
\colorbox[RGB]{ 217,210,233 }{ \texttt{ (1,0) <--> (1,1) ; (3,4) <--> (2,4) ; (4,2) <--> (4,3) ; (0,0) <--> (0,1) ; (0,3) <--> (0,2) ; }}
\colorbox[RGB]{ 217,210,233 }{ \texttt{ (4,4) <--> (3,4) ; (4,3) <--> (4,4) ; (4,1) <--> (4,2) ; (2,1) <--> (2,2) ; (1,4) <--> (0,4) ; }}
\colorbox[RGB]{ 217,210,233 }{ \texttt{ (1,2) <--> (0,2) ; (2,4) <--> (2,3) ; (4,0) <--> (3,0) ; (2,2) <--> (3,2) ; (1,2) <--> (2,2) ; }} 
\colorbox[RGB]{ 217,210,233 }{ \texttt{ (1,3) <--> (0,3) ; (3,2) <--> (3,3) ; (0,2) <--> (0,1) ; (3,1) <--> (3,2) ; (1,3) <--> (1,4) ; }}

\colorbox[RGB]{ 217,210,233 }{ \texttt{ <ADJLIST\_END> } } \colorbox[RGB]{ 217,234,211 }{ \texttt{ <ORIGIN\_START> (1,3) <ORIGIN\_END> } } \colorbox[RGB]{ 234,209,220 }{ \texttt{ <TARGET\_START> (2,3) <TARGET\_END> } } 

\colorbox[RGB]{ 207,226,243 }{ \texttt{ <PATH\_START> (1,3) (0,3) (0,2) (1,2) (2,2) (2,1) (2,0) (3,0) (4,0) (4,1) (4,2) (4,3) (4,4) }}
\colorbox[RGB]{ 207,226,243 }{ \texttt{ (3,4) (2,4) (2,3) <PATH\_END> } } \\
	\end{minipage}
	\caption{
		Example text output format with token regions highlighted.
		\colorbox[RGB]{ 217,210,233 }{Adjacency list}: text representation of the graph,
		\colorbox[RGB]{ 217,234,211 }{Origin}: starting coordinate,
		\colorbox[RGB]{ 234,209,220 }{Target}: ending coordinate,
		\colorbox[RGB]{ 207,226,243 }{Path}: maze solution sequence.
		By passing an instance of
		\href{https://understanding-search.github.io/maze-dataset/maze_dataset/tokenization.html\#MazeTokenizerModular}{\texttt{MazeTokenizerModular}}
		to
		\href{https://understanding-search.github.io/maze-dataset/maze_dataset.html\#MazeDataset.as_tokens}{\texttt{as\_tokens(...)}}
		a maze can be converted to a text sequence. The
		\href{https://understanding-search.github.io/maze-dataset/maze_dataset/tokenization.html\#MazeTokenizerModular}{\texttt{MazeTokenizerModular}}
		class contains a rich set of options with 19 discrete parameters, resulting in over 5.8 million unique possible tokenizers.
	}
	\label{fig:token-regions}
\end{figure}