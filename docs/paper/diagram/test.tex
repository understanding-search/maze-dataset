\documentclass{article}

% Basic packages
\usepackage[utf8]{inputenc}
\usepackage[T1]{fontenc}
\usepackage{graphicx}
\usepackage{amsmath,amssymb}
\usepackage{xcolor}  % Required for colors
\usepackage{transparent}  % Required for transparency
\usepackage{import}  % For importing the PDF_TEX file

% Page margins
\usepackage[margin=1in]{geometry}

% Title information
\title{Maze Visualization}
\author{Your Name}
\date{\today}

\begin{document}

\maketitle

\section{Maze Diagram}

Below is the maze representation generated from Inkscape:

\begin{figure}[h]
    \centering
    % Set the desired width of the SVG image
    \def\svgwidth{0.8\textwidth}
    % Import the PDF_TEX file
    \input{diagram-4.pdf_tex}
    \caption{Maze representation with different visualization methods}
    \label{fig:maze}
\end{figure}

\section{Notes on the Diagram}

This diagram shows various ways to visualize and represent a maze:

\begin{itemize}
    \item The top-left shows the configuration setup with \texttt{MazeDatasetConfig}
    \item The center shows how to access a specific maze with \texttt{ds[0]}
    \item The right side shows ASCII representation with \texttt{m.as\_ascii()}
    \item The bottom sections show other visualization methods:
    \begin{itemize}
        \item \texttt{m.as\_pixels()} for pixel representation
        \item \texttt{MazePlot(m)} for a graphical plot
        \item \texttt{m.as\_tokens(...)} for a tokenized representation
    \end{itemize}
\end{itemize}

\section{Using the Maze Dataset}

The diagram illustrates the typical workflow for working with the maze dataset:

\begin{enumerate}
    \item Configure the maze parameters using \texttt{MazeDatasetConfig}
    \item Create the dataset with \texttt{MazeDataset.from\_config(cfg)}
    \item Access individual maze objects from the dataset
    \item Visualize or process mazes using various methods
\end{enumerate}

\end{document}